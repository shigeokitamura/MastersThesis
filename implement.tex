\chapter{実装}
\label{chapter:proposal}
本章では,減算型表示とSearch by Snapを用いて実装したプロトタイプ及び看板認識の手法について述べる.

\section{リアルタイム看板認識APIの実装}

    \begin{table}[t]
    \caption{プロトタイプに用いたデータ(2018年12月時点OSMのデータベースを基に作成)}
    \label{tab:storelist}
    \begin{center}
      \begin{small}
        \begin{tabular}{c|cccccc}
        \hline\hline
        id & name & shop/amenity & opening\_hours & payment:visa \\ 
        \hline
        2391866925 & セブン-イレブン 高槻高槻町店 & convenience & 24/7 & yes \\
        5279265335 & 全席個室居酒屋 北海の恩返し 高槻店 & bar & 17:00-24:00 & yes \\
        5281672553 & 小だるま JR高槻駅前店 & bar & Mo-Th 11:30-14:00; \ldots & yes \\
        5281672555 & 肉丼専門店 高槻肉劇場 & restaurant & 11:00-23:00 & no \\
        5281672556 & おだいどこはなれ 高槻店 & bar & Mo-Th 11:30-24:00; \ldots & yes \\
        5281672557 & ビリヤード・ダーツ\&Food Bar Ozbuddy & pub & Mo-Th 14:00-26:00; \ldots & no \\
        5281672577 & 磯丸水産 高槻店 & restaurant & 24/7 & yes \\
        5281672578 & 炭焼酒場 森田家 高槻店 & restaurant & 17:00-26:00 & yes \\
        5281672579 & メサベルテ 高槻店 & bakery & Mo-Sa 07:00-20:30; \ldots & no \\
        5281672580 & 駿河屋 & hobby & 10:00-23:00 & no \\
        5281672581 & 甲南チケット 高槻本通店 & ticket & Mo-Sa 10:00-19:00; \ldots & no \\
        5281672739 & 焼肉・しゃぶしゃぶ食べ放題ぷくぷく 高槻店 & restaurant & Mo-Fr 16:30-25:00; \ldots & yes \\
        5281672740 & 赤から 高槻店 & bar & Mo-Fr 17:00-24:00; \ldots & yes \\
        5281672835 & ちどり亭 高槻店 & restaurant & 17:00-25:00 & yes \\
        5281672942 & 高槻ちゃぶちゃぶ & bar & 17:54-06:00 & yes \\
        \hline
        \end{tabular}
      \end{small}
    \end{center}
    \end{table}

    \begin{table}[tb]
        \caption{看板画像の学習と検出に用いたマシンのスペック}
        \label{tab:machine}
        \begin{center}
        \begin{tabular}{c|c}
            \hline\hline
            要素 & スペック \\
            \hline
            CPU & Intel (R) Core (TM) i7-8700K @ 3.70 GHz \\
            RAM & 16 GB \\
            GPU & NVIDIA GeForce GTX 1080 \\
            VRAM & 12 GB \\
            OS & Ubuntu 17.10 \\
            \hline
        \end{tabular}
        \end{center}
    \end{table}

    \subsection{対象とするデータ}
    \ref{sec:system}章で述べた構成に基づき,大阪府高槻市のJR高槻駅と阪急高槻市駅の間にある商店街の一部を対象としたプロトタイプを実装する.看板の認識には,少ない学習データから高い精度での認識を可能にするために,文献\cite{Kitamura2018}で提案したAPIを用いる.
    プロトタイプでは,図\ref{fig:map}の赤線部で示されている高槻本通の$100 m$区間においてOSM上にデータが存在する店舗のうち,15店舗を対象とする.
    対象とする店舗のノードを表\ref{tab:storelist}に示す.
    OSM上では,飲食店などの店舗のオブジェクトは位置を定義された単一の点であるノードとして扱われ,各オブジェクトには``OSM ID''という一意の番号が付与されている.
    各店舗の看板画像とOSM IDを紐付け,Overpass API\footnote{http://overpass-api.de}を用いて店舗情報を取得することにより,看板画像から店舗情報へのアクセスができる.
    表\ref{tab:storelist}内の``id''は各店舗のOSM IDを表しており,それぞれのノードには複数のタグが設定されている.
    ``name''は店舗の名称が代入されている.
    ``shop''タグは,店舗が販売している商品を記述するために使用される.
    飲食店の場合,施設の種類を表す``amenity''タグにバーやレストランのような店舗の種類が代入されている.
    ``opening\_hours''タグには,店舗の営業時間が代入されている.年中無休で24時間営業の場合は``24/7''が,曜日によって営業時間が異なる場合は,セミコロン区切りで複数の値が代入されている.例えば,小だるま JR高槻駅前店は,月曜日から木曜日までは11時30分から14時00分と17時から23時30分まで,金曜日,土曜日は11時30分から14時30分と17時00分から24時30分,日曜日は11時30分から23時30分が営業時間である.この場合,``opening\_hours''タグの値は
    ``Mo-Th 11:30--14:00, 17:00--23:30; Fr-Su 11:30--14:30, 17:00--24:30; Su 11:30--23:30''となる.
    支払いにクレジットカードが利用可能かどうかは,``american\_express'',``payment:diners\_club'',``payment:jcb'',``payment:master'',``payment:visa'',のタグに``yes''が代入されていれば利用可能である.表\ref{tab:storelist}には``payment:visa''タグのみを掲載している.

  \subsection{看板画像の学習}
    看板の認識は,まず写真の中から看板領域を抽出し,次に,切り出された看板がどの店舗のものなのかを識別するという2段階で行う.
    看板画像の学習は,文献\cite{Kitamura2018}と同一の条件下で行った.学習に用いたマシンのスペックを表\ref{tab:machine}に示す.
    写真の中から看板の領域を抽出するために,Tensorflow 1.0\cite{abadi2016tensorflow}で実装されたYOLOv2\cite{YOLO9000}を用いた.
    対象とする店舗の前で様々な角度から合計650枚の写真を撮影し,それぞれに人手で看板領域をアノテーションした上で,585枚を教師データ,65枚をテストデータとして学習させた.

    抽出した看板画像から店舗名を識別するためには,VGG16\cite{VGG16}を用いた.
    表\ref{tab:storelist}に記載した15店舗の看板画像を昼の時間帯に各店舗100枚ずつ,合計1,500枚の写真を収集し,1店舗につき教師データ50枚,バリデーションデータ25枚,テストデータ25枚に分類して学習させた.

    看板認識を行うAPIは,Python \footnote{https://www.python.org} 3.6.5とWeb APIフレームワークであるFalcon \footnote{https://falconframework.org} 1.4.1を用いて,を用いて学習に用いたマシンと同一のマシンに構築したサーバ上に構築した.




\section{評価すべき項目の検討}