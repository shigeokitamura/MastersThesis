\chapter{議論}
\label{chapter:discussion}
本章では,\ref{chapter:experiment_dr}章と\ref{chapter:experiment_sbs}章で述べた実験の結果に基づき,本研究の到達点と改善点,今後の展望について述べる.

\section{得られた知見}
\label{section:obtained_knowledge}
  \ref{chapter:experiment_dr}章で述べた実験では,探索対象が単体であり,時間帯が昼の場合,提案手法であるハイブリッド型情報提示手法は文献\cite{Fujita:2013}で提案された減算型情報提示手法と探索時間に関して有意差は見られなかった.
  しかし,探索対象が複数の場合及び時間帯が夜で探索対象が単体の場合において,提案手法は減算型情報提示手法よりも探索時間が有意に短いことが確認された.
  このことから,背景の彩度が低い場合や複数の看板を探索する場合において,提案手法は有効であると考えられる.
  また,時間帯が夜で探索対象が複数である場合,加算型情報提示手法と比較して減算型情報提示手法の方が,探索時間が長くなる傾向が見られた.
  これは実験に用いた写真の背景の彩度が低かったこと,実験に用いたビルの看板に白と黒から構成されているものが多く含まれていたことにより,減算の効果が減少したことが原因と考えられる.

  \ref{chapter:experiment_sbs}章で行った実験の結果から,情報探索時間に関しては,本稿で扱ったいずれの場合においても,提案システムを用いた場合は食べログを用いた場合よりも探索時間が有意に短くなることが明らかとなった.このことから,提案システムを用いることによって,位置情報のみを用いた場合と比較してユーザはより素早く求めている情報を取得することが可能になるといえる.
  情報探索の正確さに関しては,本稿で扱ったいずれの場合においても,提案システムを用いた場合と食べログを用いた場合とでは正解率に有意差は見られなかった.このことから,言語障壁がなければ位置情報のみを用いても正確に情報を探索することが可能であり,提案システムを用いた場合においても正確な情報探索が可能であるといえる.
  さらに,アンケート結果から,提案システムを用いることによって,位置情報のみを用いた場合と比較して,より簡単かつ直感的に情報が探索できるようになることが示唆された.

\section{本研究の到達点}
  \ref{chapter:experiment_dr}章で実施した実験では,文献\cite{Fujita:2013}で提案された減算型情報提示手法に文字情報を追加した加算型と減算型のハイブリッド型情報提示手法を提案した.提案手法を用いたシステムのプロトタイプを実装することにより,減算型情報提示手法の問題点であった,不要な情報を減算する際に対象となる看板が白黒であり,かつその周辺の景色の彩度が低い場合に減算の効果が減少するという点が解消されたと考えられる.これにより,彩度が低い環境においても分かりやすい情報提示が可能となった.また,実験結果から看板を探索する際,提案手法を用いることによって探索時間が短くなることが確認された.以上により,\ref{section:purpose}節で述べた看板密集地域における視覚情報の識別性の向上,及び探索時間の短縮が可能になった.

  提案システムを用いることによって,本研究の目的である,ユーザの目の前にある店舗の情報を直感的かつ簡単に取得できることが達成されたと考えられる.
  これにより,\ref{section:searching_action}節で述べた,慣れていない地域においても,ユーザが求める条件に合致する店舗を探索できるようになることが示唆された.

\section{改善点}
  \subsection{減算型表示の実環境における実験}
    \ref{chapter:implement_dr}章で実装したシステムは特定の位置で撮影した全天球画像内でのみ店舗の探索が行えるため,\ref{chapter:experiment_dr}章での実験から得られた知見は,人工環境内における結果にとどまる.
    そのため,実環境においても本稿で提案した情報提示手法が有用であるかを検証する必要がある.
    \ref{chapter:implement_recog}章で実装したリアルタイム看板認識APIを用いることにより,実環境での実験が可能になるため,看板が密集している地域においてOSMのデータベースと看板データベースを構築し,実環境において提案手法の優位性を検討する.

  \subsection{看板認識手法}
    \ref{chapter:implement_recog}章で実装したシステムの改善点として,(1)OSMのノードと看板画像を手作業で関連付けなければならない点,(2)インターネット上の情報から多種多様な店舗の看板画像を大量に集めることは困難であるため,手作業で看板画像を1店舗につき100枚程度集めなければならない点,が挙げられる.
    (1)に対しては,看板画像を提示してユーザに店舗名を回答するシステムを実装することで解決でき,(2)に対してはユーザに店舗の看板画像を提示し,それと同じ写真を撮影して投稿するシステムを実装することで解決できると考えられる.
    これらのシステムにゲーミフィケーションを利用し,ユーザの行動に対して報酬を与えることによって,多数のデータを効率よく収集できると考えられる.
    
    今後の展望として,看板認識をサーバ上で行うのではなく,Tiny--YOLO等を用いて携帯端末上で行うことを検討する.これにより,サーバへ画像を送信する必要がなくなるため,通信量の大幅な軽減が期待される.

  \subsection{対象地域の拡張}
    \ref{chapter:implement_sbs}章で用いた,OSMのデータは誰もが編集可能であるため,その地域に慣れている地元のユーザが自身でデータを収集し,OSMを通して活用できるようになる枠組みの構築を目指す.
    OSMのノードには``cuisine''タグが存在し,``burger'',``noodle'',``japanese'',``chinese''など,飲食店で提供される食品の種類を表す値を追加できる.他にもベジタリアン向けのメニューが提供されていることを表す``diet:vegetarian''や,イスラム教の戒律で許されている食品のみを使用したメニューが提供されていることを表す``diet:halal''などのタグも存在する.
    さらに,店舗名の英語またはローマ字表記を表す``name:en''タグや中国語表記の``name:zh''タグ,韓国語表記の``name:ko''タグなどを充実させることによって,ユーザインタフェースを多言語に対応させることが可能となる.
    これらのデータを活用することにより,その地域に慣れていない人や,地元の文字が読めない外国人観光客に対して,求めている情報を容易に取得でき,アレルギーや宗教的制約などの理由による食事制限にも対応可能なナビゲーションを実現できる.
    
    本稿における実験参加者は地元の大学生であるため,地域に慣れていないユーザや非漢字圏など地元の文字が読めないユーザを対象としたユーザ実験を実施する.