\chapter{結論}
\label{chapter:conclusion}
  本研究の目的は,ユーザが慣れていない地域や周囲の文字が読めない状況であっても,目の前にある店舗の情報を直感的かつ簡便に取得できるシステムの実現である.
  本稿では「周囲に看板が多数存在する状況」と「周囲の文字が読めない状況」を対象としたプロトタイプを実装し,ユーザ実験を行うことで,既存のシステムとの差異を明らかにした.
  以下の本稿の内容を纏める.

  \ref{chapter:introduction}章では,本研究に至った背景と問題点を述べ,解決すべき課題を明らかにした.
  街中における店舗の看板は,人々が求める条件に合致した店舗を探す際に重要な役割を果たしている.
  しかし,街中における情報探索の問題点として(1)繁華街など看板が密集している地域においては,大量に存在する他の視覚情報に紛れて目的の看板を見つけることが出来ない可能性があること,(2)外国人観光客の場合,言語障壁により目の前にある店舗が自身の求める条件に合致するか検索することが困難である点,の2点を挙げた.
  (1)に対しては,ユーザにとって不要な情報を目立たなくさせ,必要な情報には追加情報を重畳表示すること,(2)に対しては,スマートフォンのカメラを通して店舗の看板を見るとカメラ映像上に店舗の詳細情報が得られるシステムを実装することにより,これらの問題を解決すると述べた.
  これらの提案手法の優位性を検証するために,従来手法と比較したユーザ実験を行うことを述べた.

  \ref{chapter:relatedwork}章では,本研究で提案するシステムの実現に関連する研究を述べ,本研究の位置付けを明らかにした.
  まず,ARを用いたナビゲーションシステムについて説明し,ARを用いることで紙の地図を用いる場合と比較してより少ない時間で目的地まで辿り着けることについて述べた.
  次に,情報の視認性に関する研究として,解像度制御による視線誘導と,隠消現実感について述べた.
  続いて,看板認識に関する研究として,OCRを用いて看板に描かれている文字を認識する研究や,物体検出に関する研究として,CNNを用いて画像の中から物体領域を検出する研究について述べた.
  最後に,本研究の位置付けとして,Vision-based ARを用いることと,CNNを用いて看板を認識することを述べた.

  \ref{chapter:design_guidline}章では,これまでの取り組みとして,減算型表示について説明し,本研究で対象とする状況及び提案システムの要件について述べた.
  加えて,提案システムの要件を満たすための前提条件と,本稿の提案手法であるSearch by Snapについて説明し,そのインタフェースのデザインについて述べた.

  \ref{chapter:implement_dr}章では,減算型表示を用いたプロトタイプの実装について述べた.
  プロトタイプでは,ユーザはモバイルデバイスを全天球画像内で全方向に向けることができ,選択されていない種類の店舗の視覚情報を減算すると述べた.

  \ref{chapter:implement_recog}章では,リアルタイムで看板認識を行うために実装したAPIについて述べた.
  (1)YOLOv2を用いて看板領域を検出し,(2)VGG16を用いて看板を分類することにより,画像内の看板を認識することを述べた.
  リアルタイムで看板認識を行うために,GPUを搭載したマシン上にWeb APIサーバを構築し,モバイルデバイスからAPIを呼び出すことでリアルタイム認識を実現した.

  \ref{chapter:implement_sbs}章では,\ref{chapter:design_guidline}章で述べたデザイン指針に基づき,Search by Snapを用いて実装したプロトタイプについて述べた.
  \ref{chapter:implement_recog}章で実装したAPIと,OpenStreetMapのデータベースに格納されている店舗情報を用いることによって,ユーザがモバイルデバイスのカメラを通して店舗の看板を見ると,その店舗の情報が得られるシステムを実装した.

  \ref{chapter:experiment_dr}章では,\ref{chapter:implement_dr}章で実装したプロトタイプを用いて実施した評価実験について述べた.
  実験参加者には,全天球画像内で指示された看板を探索するタスクを課した.
  実験の結果,ユーザにとって不要な視覚情報を減算し,必要な情報には付加情報を重畳表示することによって,探索時間が有意に短くなることを示した.

  \ref{chapter:experiment_sbs}章では,\ref{chapter:implement_sbs}章で実装したプロトタイプを用いて実施したユーザ実験について述べた.
  実験参加者には,実際の商店街において,ユーザが条件に合う店舗を探している状況を想定し,使えるクレジットカードの種類と月曜日の営業開始時刻を調べるタスクを課した.
  提案システムと位置情報を用いた検索サービスとを比較した結果,探索時間に関しては提案システムを用いることで有意に短くなることを示し,正解率に関しては有意差がなく,どちらも正確に探索ができることを示した.

  \ref{chapter:discussion}章では,\ref{chapter:experiment_dr}章及び\ref{chapter:experiment_sbs}章で実施した実験から得られた知見と本研究の到達点,提案システムの改善点を述べた.
  本稿で提案した2つのシステムを用いることによって,ユーザが慣れていない地域においても,ユーザが求める条件に合致する店舗を探索できるようになることが示唆された.
  改善点としては,(1)\ref{chapter:implement_dr}で実装したプロトタイプを実環境でも利用できるよう新たに実装を行う点,(2)看板認識に必要なデータセットを効率よく収集する手法を確立する点,(3)OSMのデータベースの充実やインタフェースを多言語化する点,の3点が挙げられる.

  \ref{chapter:conclusion}章では,本稿の要点をまとめ結論づけた.