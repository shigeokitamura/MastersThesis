\chapter{関連研究}
\label{chapter:relatedwork}
本章では,関連研究について述べ,本研究の位置付けを明らかにする.

\section{ARを用いた情報提示に関する研究}
  \subsection{拡張現実感}
    拡張現実感(Augmented Reality; AR)とは,ユーザが見ている現実のシーンに仮想物体を重畳することで,ユーザがいる場所に応じた情報を直感的に提示する技術の総称である\cite{Kambara:2010}.近年,ARを用いたナビゲーションシステムや付加情報提示システムが多数提案されている.ARは主に,Location-based ARとVision-based ARに分類できる\cite{Chatzopoulos:2017}.本研究はVision-based ARを用いる.

  \subsection{ARを用いたナビゲーション}
    ARを用いたナビゲーションシステムに関する研究も行われている.
    Georgらは,屋内など限定された地域におけるナビゲーションシステムを提案している\cite{Gerstweiler:2018}.
    Umairらは,携帯端末上でのARを用いたナビゲーションは,紙の地図を用いた場合と比較して,より短い時間と少ない操作で目的地まで辿り着けることを示している\cite{Rehman:2017}.

    
    
  %ビジョンベースARとは,システムがカメラを通して見る画像自体を認識し,画像内にある物体を特定することで現実世界に情報を付与するARである.吉野らは,道に迷いやすい人の特徴を考慮したナビゲーションシステム``DoCoKa''の開発を行っている\cite{Yoshino2013}.これはQRコードマーカを目印として,スマートフォンの画面上にARを用いてユーザの進むべき方向の矢印を表示する.これにより,迷いやすい人でも直感的に屋内の目的地に到着できるシステムである.また,Umairらは,ウェアラブルデバイスであるGoogle Glass\footnote{\url{https://www.google.com/glass/start/}(2017/1/18確認)}を用いて,Wi-FiやBluetoothと地磁気センサにより屋内での現在位置を検出し,曲がるべき方向の矢印を提示するナビゲーションシステムを考案している\cite{Umair2015}.しかし,これらのビジョンベースARのみを用いたシステムは,屋内など特定の場所でしか利用できない.


\section{看板認識に関する研究}
  看板に書かれてある文字を認識する研究は多数行われている.
  主な手法としては,ニューラルネットワークを用いて看板に書かれている文字を認識する手法や,Optical Character Recognition(OCR)を用いる手法が挙げられる.
  Heらは,シーケンスラベリング問題として背景から文字を読み取る再帰型ニューラルネットワークを開発している\cite{He:2016}.
  Leeらは,特徴量を用いてストリートビュー画像から文字領域を検出する手法を提案し,OCRソフトウェアが文字認識することを容易にしている\cite{Lee:2016}.
  しかし,看板の中には手書き文字など崩した文字で書かれているものもあり,人間であっても読むことが容易でない場合がある.このような場合は,OCRを用いて文字を認識することは困難である.
  Kavatiらは,スマートフォンで撮影された看板や標識の写真内の文字をOCRによって認識し,英語からテルグ語に翻訳してユーザに提示する旅行者向けのWebアプリケーションを開発している\cite{Kavati:2017}.
  